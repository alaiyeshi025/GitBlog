% Created 2020-03-04 三 20:19
% Intended LaTeX compiler: pdflatex
\documentclass[11pt]{article}
\usepackage[utf8]{inputenc}
\usepackage[T1]{fontenc}
\usepackage{graphicx}
\usepackage{grffile}
\usepackage{longtable}
\usepackage{wrapfig}
\usepackage{rotating}
\usepackage[normalem]{ulem}
\usepackage{amsmath}
\usepackage{textcomp}
\usepackage{amssymb}
\usepackage{capt-of}
\usepackage{hyperref}
\usepackage{xeCJK}
\author{Johnny Zhang}
\date{\today}
\title{}
\hypersetup{
 pdfauthor={Johnny Zhang},
 pdftitle={},
 pdfkeywords={},
 pdfsubject={},
 pdfcreator={Emacs 26.3 (Org mode 9.3.6)},
 pdflang={English}}
\begin{document}

\section{说般若纲要}
\label{sec:org2c5a7e9}
\begin{enumerate}
\item 第一义,即谓本性。性为绝待之体,故曰第一义。性体空寂,故曰第一义空。
\item 凡明诸法缘生之义者,曰俗谛。凡明缘生即空之义者,曰真谛。
\item 执著分两种:执五蕴色身为我,名曰人我执,简言之,曰我执;执著一切诸法,名曰法我执,简言之,则曰法执。我执不除,生烦恼障。法执不除,生所知障。总名惑障。由惑造业,则为业障。因业受苦,名曰苦障,亦名报障。我法二执,细分之,又有分别、俱生之别。起心分别,因而执著者,为分别我法二执,故粗;并未有意分别,而执著之凡情随念俱起者,为俱生我法二执。
\end{enumerate}

\section{五重释题}
\label{sec:org169889a}
\begin{itemize}
\item 天台宗,五重:名、体、宗、用、相
\item 名:金刚般若波罗密经
\item 体:生实相
\item 宗:离一切相,修一切善
\item 用:破我,灭罪,成就如来
\item 相(判教):至圆极顿
\end{itemize}

\section{文义}
\label{sec:orgbcc13ac}
\subsection{序分}
\label{sec:orgff0515d}
\subsubsection{证信序(又名:通序,经后序,遗教序)}
\label{sec:org9760c8d}
\subsubsection{发起序}
\label{sec:org7725812}
\begin{itemize}
\item 佛为出家制三衣。一名安陀会,此名五条,亦名著体衣,作务及坐卧著之。一名郁多罗僧,此名七条,讲经说法,则加于五条之上著之,故又名上衣。若居稠人广众,或入大都会及王宫,则著九条者,梵名僧伽黎,亦名大衣。五条又名下品,七条又名中品,九条又名上品。
\item 三衣,总名袈裟,加沙者杂也
\end{itemize}
\subsection{正宗分}
\label{sec:org5c2afbe}
\subsection{流通分}
\label{sec:orgd5d1c5e}
\end{document}